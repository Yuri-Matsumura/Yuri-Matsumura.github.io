\documentclass[10pt]{article}
\usepackage{geometry}
\geometry{a4paper, margin=1in}
\usepackage{enumitem}
\usepackage{hyperref}
\setlength{\parindent}{0pt}
\setlength{\parskip}{6pt}

\begin{document}

\begin{center}
    {\LARGE \textbf{Yuri MATSUMURA}} \\
\end{center}

\vspace{1mm}
Email: \href{mailto:Yuri.Matsumura@rice.edu}{Yuri.Matsumura@rice.edu} \\
Homepage: \href{https://sites.google.com/view/yurimatsumura/}{https://sites.google.com/view/yurimatsumura/} \\
LinkedIn: \href{https://www.linkedin.com/in/yuri-matsumura-2ab0681b8}{www.linkedin.com/in/yuri-matsumura-2ab0681b8} \\
Phone: +1 (281) 777 1076

\vspace{10pt}

\textbf{RESEARCH INTERESTS} \\
Empirical Industrial Organization, Applied Economics, Competition Policy

\vspace{10pt}

\textbf{EDUCATION} \\
Ph.D. in Economics, Rice University \hfill 2019 - 2025 (expected) \\
M.A. in Economics, The University of Tokyo \hfill 2018 \\
B.A. in Economics, Osaka University \hfill 2016

\vspace{10pt}

\textbf{HONORS} \\
Dissertation Research Improvement Grants, Rice University \hfill 2023 - 2024 \\
Rice Fellowship, Rice University \hfill 2019 - 2024

\vspace{10pt}

\textbf{PUBLICATIONS} \\
“Resolving the Conflict on Conduct Parameter Estimation in Homogeneous Goods Markets between Bresnahan (1982) and Perloff and Shen (2012)”, with Suguru Otani, \textit{Economics Letters}, 2023 \\
“Dissolution Risk and Legislative Effort of Politicians”, with Yumi Koh and Ken Onishi, \textit{Electoral Studies}, 2022

\vspace{10pt}

\textbf{JOB MARKET PAPER}\\
“Evaluating Cartel Impact in Procurement Auctions”\\
Abstract: From 2018 to 2020, four incumbent firms formed a cartel in Japan's electricity retail market. During this period, they employed a market allocation scheme, withdrawing from each other's sales areas to maximize individual profits in own area. 
This paper examines the cartel's impact on procurement auctions used to select electricity suppliers for public buildings.
The analysis reveals several key findings: (1) cartel members reduced their participation rates in procurement auctions held in outside their designated areas, (2) when they did participate, they submitted non-competitive (phony) bids, and (3) the average winning bids increased during the cartel period.   
To explore a counterfactual scenario without the cartel, I estimate an auction model with asymmetric risk-averse bidders. 
The simulation shows that if cartel members had continued participating in auctions in each other's areas, winning bids would have decreased, but auction efficiency would have been distorted due to bidder asymmetry.

\vspace{10pt}

\textbf{OTHER WORKING PAPERS} \\
“Challenges in Statistically Rejecting the Perfect Competition Hypothesis Using Imperfect Competition Data”, with Suguru Otani, (Submitted to \textit{International Journal of Industrial Organization}) \\
“An MPEC Estimator for Conduct Parameter Estimation in Homogeneous Goods Markets”, with Suguru Otani

\vspace{10pt}

\textbf{WORK IN PROGRESS} \\
“A New GPV Estimator Using Bernstein Polynomial” \\
“Identification of Firm Conduct in Homogeneous Product Markets” with Suguru Otani

\vspace{10pt}

\textbf{TEACHING AND RESEARCH EXPERIENCES} \\
Teaching Assistant at Rice University: Applied Microeconomics (U), Applied Econometrics (U),\\
Mathematical Economics (U), Macroeconomics (G), Matching and Market Design (G) \hfill 2020 - 2024\\
Teaching Assistant at The University of Tokyo: Industrial Organization (U, Economics), Microeconomics (G, Graduate School of Public Policy) \hfill  2017 - 2018\\
Teaching Assistant at International Christian University: Microeconomics (U) \hfill 2018\\
Research Assistant for Professor Jeremy Fox, Rice University\hfill 2021 - 2022\\
Research Assistant for Professor Daiji Kawaguchi, The University of Tokyo \hfill 2018 - 2019\\
Research Assistant for Professor Mototsugu Shintani, The University of Tokyo \hfill 2016 - 2018

\vspace{10pt}

\textbf{ACADEMIC SERVICE} \\
Referee: \textit{Legislative Studies Quarterly}

\vspace{10pt}

\textbf{SEMINARS AND PRESENTATIONS} \\
2024: Brown Bag Workshop (Rice University) \\
2018: Microeconomics Workshop (The University of Tokyo), Japanese Economic Association, Spring meeting (University of Hyogo), Junior IO Workshop (The University of Tokyo)

\vspace{10pt}

\textbf{OTHERS} \\
Programming Skills: Julia, R, MATLAB, Python, Stata \\
Languages: English, Japanese (native)

\vspace{10pt}

\textbf{REFERENCES} \\
Professor Jeremy Fox (Advisor) \\
Department of Economics, Rice University \\
Email: \href{mailto:jeremyfox@gmail.com}{jeremyfox@gmail.com} \\

Professor Yunmi Kong \\
Department of Economics, Rice University \\
Email: \href{mailto:yunmi.kong.01@gmail.com}{yunmi.kong.01@gmail.com} \\

Professor Mallesh Pai \\
Department of Economics, Rice University \\
Email: \href{mailto:mallesh.pai@rice.edu}{mallesh.pai@rice.edu}

\end{document}
