\documentclass[10pt]{article}
\usepackage{geometry}
\geometry{a4paper, margin=1in}
\usepackage{enumitem}
\usepackage{hyperref}
\setlength{\parindent}{0pt}
\setlength{\parskip}{6pt}

\begin{document}

\begin{center}
    {\LARGE \textbf{Yuri MATSUMURA}} \\
\end{center}

\vspace{1mm}
Email: \href{mailto:Yuri.Matsumura@rice.edu}{Yuri.Matsumura@rice.edu} \\
Homepage: \href{https://sites.google.com/view/yurimatsumura/}{https://sites.google.com/view/yurimatsumura/} \\
LinkedIn: \href{https://www.linkedin.com/in/yuri-matsumura-2ab0681b8}{www.linkedin.com/in/yuri-matsumura-2ab0681b8} \\
Phone: +1 (281) 777 1076

\vspace{10pt}

\textbf{RESEARCH INTERESTS} \\
Empirical Industrial Organization, Applied Microeconomics, Competition Policy

\vspace{10pt}

\textbf{EDUCATION} \\
Ph.D. in Economics, Rice University \hfill 2019 - 2025 (expected) \\
M.A. in Economics, The University of Tokyo \hfill 2018 \\
B.A. in Economics, Osaka University \hfill 2016

\vspace{10pt}

\textbf{HONORS} \\
Dissertation Research Improvement Grants, Rice University \hfill 2023 - 2024 \\
Rice Fellowship, Rice University \hfill 2019 - 2024

\vspace{10pt}

\textbf{PUBLICATIONS} \\
“Resolving the Conflict on Conduct Parameter Estimation in Homogeneous Goods Markets between Bresnahan (1982) and Perloff and Shen (2012)”, with Suguru Otani, \textit{Economics Letters}, 2023 \\
“Dissolution Risk and Legislative Effort of Politicians”, with Yumi Koh and Ken Onishi, \textit{Electoral Studies}, 2022

\vspace{10pt}

\textbf{JOB MARKET PAPER}\\
“Evaluating Cartel Impact in Electricity Procurement”\\
Abstract:  This paper examines the impact of a market allocation cartel in the Japanese electricity retail market, active from 2018 to 2020. During this period, four incumbents restricted competition by avoiding entry into each other's regions. Analyzing electricity procurement auctions from both competitive and cartel periods, I find that cartel members reduced their participation rates and submitted complementary bids in other regions while increasing bid levels within their own regions, leading to winning bids rising by up to 9\% despite continued competition with non-cartel firms. 
Counterfactual simulations using a model of auctions with asymmetric, risk-averse bidders suggest that without the cartel, continued market entry by these firms could have lowered winning bids by up to 5.5\% and reduced winning costs by 3.4\%. 
While increased competition led to minor inefficiencies due to asymmetry among bidders, the cartel's exclusionary practices caused inefficiencies of up to 26\%.
Additionally, shifts toward nuclear energy generation were associated with lower procurement costs. 
These findings highlight the financial burden of cartel behavior and the benefits of fostering market competition for public institutions.

\vspace{10pt}

\textbf{WORK IN PROGRESS} \\
“An MPEC Estimator for Conduct Parameter Estimation in Homogeneous Goods Markets”, with Suguru Otani\\
“A New GPV Estimator Using Bernstein Polynomial” \\
“Identification of Firm Conduct in Homogeneous Product Markets” with Suguru Otani

\vspace{10pt}

\textbf{TEACHING AND RESEARCH EXPERIENCES} \\
Teaching Assistant at Rice University: Applied Microeconomics (U), Applied Econometrics (U),\\
Mathematical Economics (U), Macroeconomics (G), Matching and Market Design (G) \hfill 2020 - 2024\\
Teaching Assistant at The University of Tokyo: Industrial Organization (U, Economics), Microeconomics (G, Graduate School of Public Policy) \hfill  2017 - 2018\\
Teaching Assistant at International Christian University: Microeconomics (U) \hfill 2018\\
Research Assistant for Professor Jeremy Fox, Rice University\hfill 2021 - 2022\\
Research Assistant for Professor Daiji Kawaguchi, The University of Tokyo \hfill 2018 - 2019\\
Research Assistant for Professor Mototsugu Shintani, The University of Tokyo \hfill 2016 - 2018

\vspace{10pt}

\textbf{ACADEMIC SERVICE} \\
Referee: \textit{Legislative Studies Quarterly}

\vspace{10pt}

\textbf{SEMINARS AND PRESENTATIONS} \\
2024: Brown Bag Workshop (Rice University) \\
2018: Microeconomics Workshop (The University of Tokyo), Japanese Economic Association, Spring meeting (University of Hyogo), Junior IO Workshop (The University of Tokyo)

\vspace{10pt}

\textbf{OTHERS} \\
Programming Skills: Julia, R, MATLAB, Python, Stata \\
Languages: English, Japanese (native)

\vspace{10pt}

\textbf{REFERENCES} \\
Professor Jeremy Fox (Advisor) \\
Department of Economics, Rice University \\
Email: \href{mailto:jeremyfox@gmail.com}{jeremyfox@gmail.com} \\

Professor Yunmi Kong \\
Department of Economics, Rice University \\
Email: \href{mailto:yunmi.kong.01@gmail.com}{yunmi.kong.01@gmail.com} \\

Professor Mallesh Pai \\
Department of Economics, Rice University \\
Email: \href{mailto:mallesh.pai@rice.edu}{mallesh.pai@rice.edu}

\end{document}
